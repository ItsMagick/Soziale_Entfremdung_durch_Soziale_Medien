%! Author = charon
%! Date = 11/20/24
\begin{abstract}
    Diese Untersuchung befasst sich mit den psychologischen, neurobiologischen und strukturellen Effekten der Nutzung
    sozialer Medien auf das Individuum sowie auf die Gesellschaft.
    Auf der Grundlage empirischer Studien wird die Argumentation dargelegt, dass algorithmusgestützte Plattformen soziale
    Interaktionen intensivieren und gleichzeitig psychische Belastungen, Abhängigkeiten sowie strukturelle Entfremdung fördern.
    Hierbei wird die Bedeutung von emotionaler Manipulation, dopaminerger Rückkopplungsschleifen und Recommender-Systemen
    hervorgehoben.
    Zum Schluss werden interdisziplinäre Gegenmaßnahmen betrachtet.
\end{abstract}