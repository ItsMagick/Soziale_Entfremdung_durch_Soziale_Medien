%! Author = charon
%! Date = 11/21/24

\section{Hintergrund}\label{sec:hintergrund}
\subsection{Soziale Medien im neurokulturellen Kontext}\label{subsec:soziale-medien-im-neurokulturellen-kontext}

Soziale Medien sind längst integraler Bestandteil des westlich geprägten Alltags.
Plattformen wie Facebook, Instagram oder TikTok dienen dabei nicht nur der Informationsverbreitung, sondern zunehmend der
sozialen Selbstverortung.
Während sie das Potenzial haben, soziale Verbindung zu stärken, zeigen Studien, dass sie gleichzeitig das Gefühl von
Einsamkeit und psychischer Belastung intensivieren~\cite{siddiq_social_2024,santini_social_2024}.

Besonders kritisch ist die algorithmische Steuerung sozialer Medieninhalte: Engagement-basierte Recommender-Systeme~\cite{milli_engagement_2024}
priorisieren emotional aufgeladene Inhalte mit hoher Interaktionswahrscheinlichkeit – ein Mechanismus, der systematisch
negative Emotionen, Empörung und soziale Polarisierung verstärkt.
Diese Architektur ist eng mit kapitalistischen Aufmerksamkeitsökonomien verbunden, in denen Aufmerksamkeit zur zentralen
Ressource wird~\cite{rey_alienation_2012}.

Neurobiologisch betrachtet greifen diese Systeme in dopaminerge Belohnungskreise ein – ähnlich wie bei klassischen
Suchtmitteln~\cite{de_social_nodate}.
Die Kombination aus kognitiver Beeinflussung, neurochemischer Konditionierung und sozialer Vergleichsdynamik erzeugt
eine hochgradig wirksame digitale Abhängigkeit.

\subsection{Sucht und emotionale Ansteckung}\label{subsec:sucht-und-emotionale-ansteckung}

Psychologische Studien zeigen, dass die Nutzung sozialer Medien mit erhöhten Raten an Depression, Einsamkeit und sozialer 
Isolation einhergeht~\cite{santini_social_2024}.
Diese Symptome folgen wiederholbaren Mustern der algorithmischen Verstärkung: Inhalte mit negativer oder feindseliger
Valenz werden wahrscheinlicher ausgespielt, da sie mehr Engagement hervorrufen~\cite{milli_engagement_2024}.
Die Folge ist eine algorithmisch erzeugte \enquote{Filterblase der Negativität}.

Kramer et al.~\cite{doi:10.1073/pnas.1320040111} konnten in einem groß angelegten Facebook-Experiment mit über 689.000
Nutzenden nachweisen, dass emotionale Ansteckung auch rein textbasiert möglich ist.
Wenn die Sichtbarkeit positiver Posts reduziert wurde, sank die eigene positive Ausdrucksweise signifikant, während die
negative stieg – und umgekehrt.
Dies ist ein Beleg dafür, dass Emotionen auch ohne direkte soziale Interaktion oder nonverbale Hinweise massenhaft über
soziale Netzwerke übertragen werden können.

Auf psychologischer Ebene wirkt somit ein doppelter Verstärkungsmechanismus.
Zum einen wird durch algorithmisch kuratierte Inhalte die Wahrnehmung der Welt systematisch negativ verzerrt, zum anderen
wird durch emotionale Ansteckung eine kollektive Affektverschiebung begünstigt, die sich auch offline niederschlagen
kann.

