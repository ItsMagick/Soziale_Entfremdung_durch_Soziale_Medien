%! Author = charon
%! Date = 11/21/24

\section{Interdisziplinäre Gegenstrategien}\label{sec:interdisziplinaere-gegenstrategien}
\begin{table*}[!t]
    \begin{tabular}{ll}
        \toprule
        \textbf{Ebene} & \textbf{Wirkung} \\
        \midrule
        Neurobiologisch & Suchtmechanismen durch Dopamin und Feedback-Loops \\
        Technisch & Verstärkung affektiver, spaltender Inhalte \\
        Psychologisch & Selbstentfremdung, Isolation trotz Interaktion \\
        Gesellschaftlich & Polarisierung, Fragmentierung sozialer Realität \\
        Ethisch & Plattforminteressen kontra Nutzerwohl \\
        \bottomrule
    \end{tabular}
    \caption{Übersicht der Wirkung von Social Media Algorithmen in verschiedenen Disziplnen}
    \label{tab:fazit}
\end{table*}

Die in Tabelle~\ref{tab:fazit} dargestellten Effekte sozialer Medien lassen sich interdisziplinär auf fünf Wirkungsebenen
aufteilen – neurobiologisch, technisch, psychologisch, gesellschaftlich und ethisch.

Die dopaminerge Verstärkung durch algorithmische Feedbackschleifen~\cite{de_social_nodate} kann durch bewusste
Unterbrechungsmuster wie \enquote{Digital Detox}, \enquote{App-Timer} oder dopaminfreie Nutzungsphasen abgemildert werden.
Besonders bei Jugendlichen sollten verhaltenspräventive Maßnahmen wie Aufklärung über digitale Reize sowie
Achtsamkeitstrainings an Schulen etabliert werden.

Milli et al.\ ~\cite{milli_engagement_2024} zeigten, dass Recommender-Systeme affektgeladenen, polarisierenden Content
systematisch bevorzugen.
Dem lässt sich durch den Einsatz von stated-preference-basierten oder \enquote{System-2-Recommendern} entgegenwirken, wie sie
von Agarwal et al.\ ~\cite{agarwal_system-2_2024} diskutiert werden. 
Diese priorisieren reflektierte statt impulsiver Entscheidungen und fördern somit kognitiv durchdachtes Engagement.

Die durch algorithmisch vermittelten Social-Comparison-Effekte verstärkte emotionale Belastung~\cite{santini_social_2024}
kann durch psychologische Resilienzförderung sowie Medienbildung abgebaut werden.
Programme sollten den Mechanismus emotionaler Manipulation~\cite{doi:10.1073/pnas.1320040111} und die Bedeutung algorithmischer
Verzerrung für das Selbstbild thematisieren.

Da algorithmische Systeme Ingroup-/Outgroup-Feindseligkeiten fördern~\cite{milli_engagement_2024}, sind neue
Plattformfunktionen erforderlich: Dazu zählen z.B.\ Slow Commenting, kontextualisierte Informationseinblendungen und
transparente Moderationslogiken.
Plattformen sollten deliberative Formate ermöglichen, die Diskurs statt Spaltung fördern.

Rey~\cite{rey_alienation_2012} argumentiert, dass Plattformen die soziale Spontaneität in kapitalistische Verwertung überführen.
Eine ethische Antwort liegt in der Regulierung der Recommender-Logik: Open-Source-Systeme, auditierbare KI-Komponenten
sowie öffentliche Alternativen zu kommerziellen Netzwerken wären erste Schritte.
Jia et al.\ ~\cite{jia_embedding_2024} schlagen zusätzlich objective-aware ranking models vor, bei denen Gemeinwohlziele
ins algorithmische Design integriert werden.

\section{Konklusion}\label{sec:konklusion}

Die Analyse zeigt, dass die Wirkung sozialer Medien nicht monokausal verstanden werden kann.
Vielmehr handelt es sich um ein multidimensionales Wirkungsgefüge, in dem psychische, neuronale, soziale und algorithmische
Dynamiken miteinander verschränkt sind.
Die Empire belegt, dass soziale Medien unter der derzeit dominanten Logik der Engagement-Optimierung nicht nur bestehende
soziale Spannungen verstärken, sondern auch emotionale Selbststeuerung untergraben und Nutzer:innen in affektive Schleifen
führen, die zur langfristigen Abhängigkeit führen können~\cite{milli_engagement_2024,santini_social_2024,de_social_nodate}.

Die im Paper vorgestellten Gegenstrategien orientieren sich explizit an den fünf analysierten Wirkungsebenen.
Dabei zeigt sich, dass weder technische Optimierung noch individuelle Resilienzbildung für sich allein ausreichen.
Nur ein intersektorales Maßnahmenbündel aus algorithmischer Transparenz, regulatorischer Kontrolle, struktureller
Plattformverantwortung und frühzeitiger Bildungsintervention kann der digitalen Entfremdung wirksam begegnen.

Soziale Medien befinden sich an einem Wendepunkt: Entweder bleiben sie Werkzeuge der Aufmerksamkeitsextraktion –
oder sie werden neu gestaltet als Infrastrukturen gesellschaftlicher Integrität.
Die wissenschaftliche Auseinandersetzung mit ihrer Wirkung ist daher nicht optional, sondern notwendige Grundlage demokratischer
Technikgestaltung.
