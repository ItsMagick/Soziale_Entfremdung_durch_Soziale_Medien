%! Author = charon
%! Date = 11/20/24

\section{Einführung}\label{sec:introduction}

Digitale Plattformen wie Instagram, TikTok oder Facebook sind längst nicht mehr nur Kommunikationsmedien, sondern
strukturierende Infrastrukturen des sozialen Alltags.
Ihre Verbreitung hat neue Räume für Interaktion, Identitätskonstruktion und politische Auseinandersetzung geschaffen –
und zugleich tiefgreifende Risiken für das psychosoziale Wohlbefinden hervorgebracht.
Während Plattformen konnektivistische Narrative propagieren, dokumentiert die Forschung zunehmend ihre Rolle als Vektoren
digitaler Entfremdung~\cite{rey_alienation_2012,milli_engagement_2024,santini_social_2024}.

Insbesondere die algorithmische Architektur sozialer Medien steht im Verdacht, sowohl emotionale Instabilität zu verstärken
~\cite{doi:10.1073/pnas.1320040111} als auch Verhaltensmuster zu fördern, die klassischen Suchtprozessen ähneln~\cite{de_social_nodate}.
Die Verknüpfung aus psychologischer Vulnerabilität, neurobiologischer Verstärkung und wirtschaftlich motivierter Reizmaximierung
hat eine Dynamik erzeugt, in der Nutzer:innen zunehmend die Kontrolle über ihr digitales Verhalten verlieren.

Ziel dieses Beitrags ist es, auf Grundlage aktueller empirischer Forschung eine interdisziplinäre Systematik zu entwickeln,
die zeigt, wie soziale Medien auf fünf zentralen Ebenen wirken: neurobiologisch, technisch, psychologisch, gesellschaftlich
und ethisch.
Die so strukturierte Analyse soll nicht nur pathologische Mechanismen sichtbar machen, sondern differenzierte Gegenstrategien
ableiten, die sowohl technologisch als auch bildungspolitisch und regulatorisch greifen.
