%! Author = chaorn
%! Date = 07.07.25

\section{Digitale Sucht und jugendliche Vulnerabilität}\label{sec:digitale-sucht-und-jugendliche-vulnerabilitat}

Neurophysiologisch aktiviert die Nutzung sozialer Medien dasselbe dopaminerge Belohnungssystem wie stoffgebundene 
Suchtmittel~\cite{de_social_nodate}.
Zu den betroffenen Hirnregionen zählen u.a.\ der Nucleus accumbens, die Amygdala sowie der präfrontale Cortex, der
für Selbstkontrolle und Entscheidungsverhalten zuständig ist.
Insbesondere Jugendliche sind durch ihre noch unreife Exekutivfunktion für diese digitalen Reizsysteme anfällig.

Das Recommender-System selbst agiert dabei wie ein konditionierender Verstärker.
Belohnung (z.B.\ Likes) erzeugt Dopaminfreisetzung.
Dadurch wird das Belohnungssystem des Gehirns konditioniert und eine Erwartungshaltung geschaffen, sodass eine Erwartungshaltung
zur nächsten Nutzung besteht.
Die erneute Nutzung sozialer Medien triggert genau diese Erwartungshaltung und setzt Dopamin erneut frei.
Hierdurch entsteht ein Feedback-Zyklus, welcher das Konsumverhalten anregt und eine endlose Rückkopplung des
Dopaminsystems manifestiert und somit die Grundlage einer Sucht schafft~\cite{doi:10.1073/pnas.1320040111}.
Dieser Prozess erklärt auch die von Santini et al.~\cite{santini_social_2024} beobachtete hohe Korrelation zwischen sozialer Medienabhängigkeit
und Einsamkeit, Depression sowie geringem Selbstwertgefühl.

Die durch Medienvergleiche ausgelöste emotionale Dysregulation~\cite{doi:10.1177/001872675400700202} ist dabei
nicht nur ein Begleiteffekt, sondern ein zentraler Wirkmechanismus der Plattformbindung. 
Kramer et al.\ ~\cite{doi:10.1073/pnas.1320040111} sprechen hier von \enquote{emotional contagion without awareness} –
also emotionaler Fremdsteuerung ohne bewusste Wahrnehmung.


