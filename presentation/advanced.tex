%! Author = chaorn
%! Date = 11.12.24

\begin{frame}{Entfremdung im digitalen Kapitalismus}
    \only<1>{
        \begin{itemize}
            \item Nutzern sind gleichzeitig Konsumenten und Produzenten (Prosumption)
            \item Die Plattformen monetarisieren \enquote{kostenlose Arbeit}
        \end{itemize}
    }
    \only<2->{
        \begin{figure}
            \centering
            \includegraphics[width=\textwidth]{res/slavery}
            \caption{https://knowyourmeme.com/photos/1630089-well-that-sounds-like-slavery-with-extra-steps}
            \label{fig:prosumption}
        \end{figure}
    }
\end{frame}

\begin{frame}{Emotionale Manipulation und Polarisierung}
    \begin{itemize}
    \item Wut, Angst, Outgroup-Feindseligkeit $\rightarrow$ höhere Verbreitung
    \item System-1-Reaktion wird getriggert (schnelle Instinkte)
    \item Realweltliche Folge: Polarisierung und soziale Fragmentierung
    \end{itemize}
\end{frame}

\begin{frame}{Entkopplung con Präferenz und Zufriedenheit~\cite{milli_engagement_2024}}
    \begin{itemize}
        \item Nutzer konsumieren algorithmisch priorisierte Inhalte
        \item Bewerten diese Inhalte aber weniger positiv als chronologische
        \item Nutzung != Zufriedenheit
    \end{itemize}
\end{frame}

\begin{frame}{Struturelle Entfremdung~\cite{rey_alienation_2012}}
    \begin{itemize}
        \item Soziale Medien = Schnittstelle für kapitalistische Verwertung
        \item Nutzer produzieren unbezahlt Aufmerksamkeit
        \item Identität wird \enquote{kuratiert} statt gelebt
        \item Entfremdung durch algorithmische Selbstinszenierung
    \end{itemize}
\end{frame}
\begin{frame}{Gesamtwirkung von Algorithmen}
    \begin{table}
        \begin{tabular}{@{}p{3.5cm}p{6.5cm}@{}}
            \toprule
            \textbf{Ebene} & \textbf{Wirkung} \\
            \midrule
            Neurobiologisch & Suchtmechanismen durch Dopamin und Feedback-Loops \\
            Technisch & Verstärkung affektiver, spaltender Inhalte \\
            Psychologisch & Selbstentfremdung, Isolation trotz Interaktion \\
            Gesellschaftlich & Polarisierung, Fragmentierung sozialer Realität \\
            Ethisch & Plattforminteressen kontra Nutzerwohl \\
            \bottomrule
        \end{tabular}
        \caption{Übersicht der Wirkung von Social Media Algorithmen in verschiedenen Disziplnen}
        \label{tab:fazit}
    \end{table}
\end{frame}

\begin{frame}{Mögliche Gegenstrategien}
    \begin{itemize}
        \item Umstellung auf \enquote{Stated-Preference-Recommender} statt Engagement-Zielsetzung~\cite{milli_engagement_2024}
        \item Entwicklung von \enquote{System-2-Recommendern} zur Förderung reflektierter Entscheidungen~\cite{milli_engagement_2024}
        \item Medienkompetenz und digitale Hygiene~\cite{santini_social_2024}
        \item Soziale Interventionen: Aufbau von Offline-Communitys, psychologische Beratung, staatliche Regulierung~\cite{siddiq_social_2024}
        \item Erhöhung algorithmischer Transparenz und Fairness
    \end{itemize}
\end{frame}