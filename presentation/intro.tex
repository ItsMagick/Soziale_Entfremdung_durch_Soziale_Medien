%! Author = charon
%! Date = 07/03/2025

\section{Einführung}\label{sec:einfuhrung}
\begin{frame}{Was ist soziale Entfremdung?~\cite{rey_alienation_2012}}
    \begin{itemize}
        \item \textit{Marxistische Entfremdung}: Trennung vom Selbst, der Arbeit, den Mitmenschen
        \item \textit{Moderne Entfremdung}: Gefühlte \enquote{Verbundenheit} bei realer sozialer Isolation
    \end{itemize}
\end{frame}
\begin{frame}{Ältere Erwachsene und soziale Medien~\cite{siddiq_social_2024}}
    \begin{itemize}
        \item Ältere Erwachsene nutzen Social Media, um mit Freunden und Familie in Kontakt zu bleiben
        \item Soziale Medien können Einsamkeit verstärken (jedoch nicht verringern), wenn sie echte soziale Interaktionen ersetzen
        \item Mangel an persönlicher Interaktion kann zu Entfremdung führen
        \item Einsamkeit = 23-fach höheres Risiko für psychische Belastung
        \item SM-Nutzung puffert diesen Effekt nicht ab
        \item Besonders betroffen: asiatische ältere Bevölkerung
    \end{itemize}
\end{frame}
\begin{frame}{Berufstätige und soziale Medien~\cite{santini_social_2024}}
    \begin{itemize}
        \item \SI{2,3}{\percent} zeigten soziale Mediensucht
        \item Direkte korrelation mit:
        \begin{itemize}
            \item Depression (odds ratio: 2.71)
            \item Einsamkeit (odds ratio: 4.4)
            \item Soziales Netzwerk (Persönliche Beziehungen)
        \end{itemize}
        \item Bi-direktionale Wirkung: Einsamkeit → Sucht → mehr Einsamkeit
    \end{itemize}
\end{frame}
\begin{frame}{Rückkopplungsmechanismus~\cite{de_social_nodate}}
    \begin{itemize}
        \item Schlechter psychologischer Zustand führt zu mehr Medienkonsum (Vermeidung, Dopaminbelohnung, Gewohnheiten)
        \item Erhöhter Medienkonsum verstärkt Einsamkeit und Depression
        \item Aktivierung des Belohnungssystems (Dopamin)
        \item Suchtmuster wie bei Substanzabhängigkeit
        \item Symptome: Toleranz, Reizbarkeit, Kontrollverlust
    \end{itemize}
    \only<2->{
        \centering \textbf{$\Rightarrow$ \alert{Stress-Induced Reversion to Habit Mechannismus triggert das impulsive System und hebelt das reflektive System aus}}~\cite{dual_process_theories_habits_2014}
    }
\end{frame}
\begin{frame}{Ambivalenz der sozialen Medien~\cite{santini_social_2024}}
    \begin{itemize}
        \item Höhere Nutzung sozialer Medien kann zur besseren Schlafqualität führen
        \item Soziale Medien können soziale Unterstützung bieten, die Einsamkeit verringert
    \end{itemize}
    \only<2->{
        \centering \Large \textbf{$\Rightarrow$ \alert{ABER: nur bei bereits geringer psychologischer Belastung!}}~\cite{shiraly_mediating_2024}
    }
\end{frame}

